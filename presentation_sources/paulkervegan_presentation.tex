\documentclass{beamer}

\usepackage[utf8]{inputenc}
\usepackage[T1]{fontenc}

\title{Modélisation, enrichissement sémantique et diffusion d'un corpus textuel semi-structuré: le cas des catalogues de vente de manuscrits.}
\date{25 septembre 2022}
\author[Paul H. Kervegan]{Paul, Hector Kervegan}

\usetheme{hector}

\begin{document}

\begin{frame}
	\titlepage
\end{frame}

\begin{frame} 
	\frametitle{There Is No Largest Prime Number} 
	\framesubtitle{The proof uses \textit{reductio ad absurdum}.} 
	\begin{theorem}
		There is no largest prime number. 
	\end{theorem} 
	\begin{enumerate} 
		\item<1-| alert@1> Suppose $p$ were the largest prime number. 
		\item<2-> Let $q$ be the product of the first $p$ numbers. 
		\item<3-> Then $q+1$ is not divisible by any of them. 
		\item<1-> But $q + 1$ is greater than $1$, thus divisible by some prime
		number not in the first $p$ numbers.
	\end{enumerate}
\end{frame}


\end{document}
