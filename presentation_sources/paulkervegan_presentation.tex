\documentclass{beamer}

\usepackage[utf8]{inputenc}
\usepackage[T1]{fontenc}
\usepackage{csquotes}

\title{Modélisation, enrichissement sémantique et diffusion d'un corpus textuel semi-structuré: le cas des catalogues de vente de manuscrits.}
\date{25 septembre 2022}
\author[Paul H. Kervegan]{Paul, Hector Kervegan}

\definecolor{wildstrawberry}{rgb}{1.0, 0.26, 0.64}

\usetheme{hector}

\usepackage{hyperref}
\hypersetup{
	colorlinks=true,
	linkcolor=wildstrawberry,
	filecolor=wildstrawberry,
	citecolor=wildstrawberry,      
	urlcolor=wildstrawberry,
	pdfauthor={Paul, Hector KERVEGAN}, 
	pdftitle={Modélisation, enrichissement sémantique et diffusion d'un corpus textuel semi-structuré: le cas des catalogues de vente de manuscrits (soutenance)}, 
	pdfsubject={Traitement de données textuelles},
	pdfkeywords={catalogues de vente}{mss}{katabase}{ocr}{web sémantique}{linked open data}{web de données}{traitement automatisé du language}{détection de motifs}{api}{fair}{rest}
}

\begin{document}

\begin{frame}
	\titlepage
\end{frame}

\begin{frame}
	\frametitle{Introduction}
	\framesubtitle{Problématique}
	
\end{frame}

\begin{frame}
	\frametitle{Introduction}
	\framesubtitle{Plan}
	\tableofcontents
\end{frame}


\section{La structure du texte comme méthode d'approche}
\subsection{Structure, document physique et encodage numérique}
\begin{frame}
	\frametitle{La structure du texte comme méthode d'approche}
	\framesubtitle{Structure, document physique et encodage numérique}
	
\end{frame}

\subsection{La spécificité d'un corpus \enquote{semi-structuré}}
\begin{frame}
	\frametitle{La structure du texte comme méthode d'approche}
	\framesubtitle{La spécificité d'un corpus \enquote{semi-structuré}}
	
\end{frame}

\section{Sous quels angles aborder cette problématique?}
\subsection{Modéliser un corpus semi-structuré}
\begin{frame}
	\frametitle{Sous quels angles aborder cette problématique?}
	\framesubtitle{Modéliser un corpus semi-structuré}

\end{frame}

\subsection{Analyser le texte à partir de sa structure: la résolution d'entités nommées}
\begin{frame}
	\frametitle{Sous quels angles aborder cette problématique?}
	\framesubtitle{Analyser le texte à partir de sa structure}
	
\end{frame}

\subsection{Recomposer et diffuser le texte via une API}
\begin{frame}
	\frametitle{Sous quels angles aborder cette problématique?}
	\framesubtitle{Recomposer et diffuser le texte via une API}
	
\end{frame}


\section{Le traitement automatique du texte comme chaîne éditoriale continue}
\subsection{L'enrichissement progressif du texte}
\begin{frame}
	\frametitle{Une chaîne éditoriale continue}
	\framesubtitle{L'enrichissement progressif du texte}
	
\end{frame}

\subsection{La résolution d'entités nommées: mettre le texte en réseau}
\begin{frame}
	\frametitle{Une chaîne éditoriale continue}
	\framesubtitle{La résolution d'entités nommées: mettre le texte en réseau}
	
\end{frame}

\subsection{Recomposer le texte et créer des documents via une API}
\begin{frame}
	\frametitle{Une chaîne éditoriale continue}
	\framesubtitle{Recomposer le texte et créer des documents via une API}
	
\end{frame}

\begin{frame}
	\frametitle{Conclusion}
\end{frame}


\begin{frame} 
	\frametitle{There Is No Largest Prime Number}
	\framesubtitle{The proof uses \textit{reductio ad absurdum}.} 
	\begin{theorem}
		There is no largest prime number. 
	\end{theorem} 
	\begin{enumerate} 
		\item<1-| alert@1> Suppose $p$ were the largest prime number. 
		\item<2-> Let $q$ be the product of the first $p$ numbers. 
		\item<3-> Then $q+1$ is not divisible by any of them. 
		\item<1-> But $q + 1$ is greater than $1$, thus divisible by some prime
		number not in the first $p$ numbers.
	\end{enumerate}
\end{frame}


\end{document}
