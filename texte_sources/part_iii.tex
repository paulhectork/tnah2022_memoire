%%%%%%%%%%%%%%%%%%%%%%%%%%%%%%%%%%%%%%%%%%%%%%%%%%%%%%%%%%%%%%
%%%%%%%%%%%%%%%%%%%%%m TROISIEME PARTIE %%%%%%%%%%%%%%%%%%%%%%
%%%%%%%%%%%%%%%%%%%%%%%%%%%%%%%%%%%%%%%%%%%%%%%%%%%%%%%%%%%%%%

\part{Rendre la recherche réutilisable et interopérable: \textit{KatAPI}, une API pour échanger des données structurées}

La problématique de cette partie est la suivante: comment rendre la recherche en humanités numériques réutilisable et encourager le partage de données entre projets de recherche? La réponse à cette question, au sein du projet \mssktb{} prend la forme d'une \api{}. Celle-ci diffuse des données source du projet, mais aussi des informations issues de la recherche et du traitement computationnel de celles-ci dans deux formats structurés: le \json{} et le \xmltei{}.

\chapter{Standards de design et statut des API dans pour les humanités numériques centrées sur le texte}
\chaptermark{Standards de design et statut}
Ici est présenté un état de l'art sur la conception d'\api{} dans un contexte d'humanités numériques. Sont présentés un standard de design pour la conception d'api (le REST), l'architecture API-DTS (équivalent du IIIF pour le texte). Si l'\api{} conçue ne se rattache pas à ces concepts, ceux-ci ont servi d'inspiration pour l'architecture de \textit{KatAPI}. Les principes FAIR (essentiels pour la recherche ouverte et interopérable) sont également présentés, ainsi que l'état de l'art pour le design d'\api{} dans un contexte d'humanités numériques. Il apparaît que de tels outils sont majoritairement conçus par des institutions (\textit{DataBnF}...) plutôt que par de simples projets de recherche.

\section{Pourquoi partager les données de la recherche? Les principes FAIR}
Les principes \gls{fair}\footcite{boeckhout_fair_2018}, qui visent à établir un ensemble de caractéristiques que les données ouvertes doivent partager, sont présentés ici.

\section{Le standard REST: un modèle pour la conception d'API}
Ici est présenté le standard REST, défini par Roy Fielding\footcite{fielding_architectural_2000}; ce standard architectural est considéré comme l'idéal à atteindre en matière de design d'\api{}. Il n'a cependant pas été suivi ici, en partie parce qu'il est mieux adapté à un projet de plus grande échelle, et en partie car il ne fonctionne pas de façon optimale avec le modèle de données défini, et notamment avec le choix de construire une \api{} renvoyant des réponses conformes à la \tei{}.

\section{CTS, OAI-PMH et DTS: quels standards pour le partage du texte en humanités numériques?}
Les principes REST, présentés ci-dessus, concernent l'architecture et le design d'une \api{}; ils ne définissent pas précisément quels formats de réponse définir, ni comment structurer ses (méta-)données. Ici sont présentés quelques protocoles visant à uniformiser les formats de réponse et l'interaction avec des API. Ils visent à encourager la standardisation et l'interopérabilité. Trois standards sont présentés en détail: le \gls{cts}\footcite{almas_continuous_2018}, \gls{oaipmh}\footcite{prime-claverie_defi_2017} et \gls{dts}\footcite{almas_distributed_2021}. Comme le REST, ces standards sont difficiles à intégrer à un petit projet de recherche; ils peuvent cependant servir d'inspiration pour la conception  d'une \api{} centrée sur le texte.

\chapter{Définir un périmètre: que partager, et comment partager?}
\chaptermark{Définir un périmètre}
Après avoir présenté le paysage des \api{} en humanités numériques, ce chapitre décrit \textit{KatAPI} du point de vue de l'utilisateur.ice: quelles données peuvent être obtenues via l'\api{}, quels paramètres de recherche sont autorisés et quels sont les formats de données retournés par l'application. C'est donc les principes de la communication du client avec l'\api{} qui sont ici présentés.

\section{Quelles données partager?}
Ici est défini le périmètre des données accessibles depuis l'\api{}: tout ce qui est produit par le projet n'est bien sûr pas accessible, et des choix ont dû être faits. Les jeux de données accessibles sont présentés ici.

\section{Codifier l'accès aux données: présentation des paramètres de recherche possibles}
Ici sont présentés les paramètres de recherche qui peuvent être utilisés pour récupérer des données, et comment ils permettent de filtrer ce qui est retourné par l'\api{}.

\section{Comment partager les données? Principes suivis pour le partage d'informations, formats et structure des réponses de l'API}
Cette section décrit les formats de réponse de l'\api{}, en \json{} et \xmltei{}, pour les différents paramètres de requête définis.

\chapter{Implémentation et fonctionnement technique de \textit{KatAPI}}
\chaptermark{Implémentation et fonctionnement technique}
Ce chapitre décrit le fonctionnement de l'\api{} côté serveur: comment les données sont reçues, la manière dont elles sont traitées et comment les réponses sont construites (notamment la création automatisée de documents \xmltei{}). Le système de gestion des erreurs est également présenté.

\section{Présentation générale}
Ici est présenté, schéma à l'appui, le fonctionnement technique de l'\api{} et la chaîne de traitement depuis la réception des requêtes jusqu'à l'envoi d'une réponse.

\section{Gestion des erreurs}
L'\api{} étant une application devant interagir avec des utilisateur.ice.s et gérer leurs requêtes, il n'est pas pensable qu'elle \enquote{plante} ou ne renvoie pas d'erreur. Tout un système de gestion des erreurs a donc été défini. Ces erreurs peuvent être classées en deux catégoriés: côté client (dues à des entrées invalides des utilisateur.ice.s)  et côté serveur (erreurs inattendues qui empêchent l'exécution d'une requête). Si ces erreurs sont rencontrées, des réponses sont construites en \json{} ou \xmltei{} (selon le format demandé par l'utilisateur.ice.s). Celles-ci décrivent l'erreur et permettent donc aux utilisateur.ice.s de corriger les erreurs qui ont pu avoir lieu côté client.

\section{Garantir le bon fonctionnement de l'application}
De la même manière que les systèmes de gestion des erreurs garantissent que l'application continue à fonctionner même en cas d'erreur, les tests garantissent que l'\api{} soit fonctionnelle, même dans \enquote{des conditions extrêmes}. Les réponses, formats et données retournées sont donc testés. C'est ce protocole de test qui est présenté ici.