\section*{Tagged Data - level 3}
\addcontentsline{toc}{section}{Tagged Data - level 3}

This repository contains digitised manuscripts sale catalogs encoded in XML-TEI at level 3.

The data have been cleaned (\href{https://github.com/katabase/2\_CleanedData}{level 2}) and post-processed (level 3).
\section*{Schema}
\addcontentsline{toc}{subsection}{Schema}

You can find the ODD that validates the encoding in the repository \href{https://github.com/katabase/Data\_extraction/tree/master/\_schemas}{Data\_extraction (folder \texttt{\_schemas})}.
\section*{Workflow}
\addcontentsline{toc}{subsection}{Workflow}

Once the data have been cleaned and post-processed, we can check them. Some errors may appear and some corrections may be needed. 

From this data, \texttt{extractor-json.py} extracts informations and retrieves them in an JSON file, \href{https://github.com/katabase/3\_TaggedData/tree/main/output}{available here}.

The script transforms this 

\begin{listing}[h!]
   \begin{minted}{xml}
<item n="80" xml:id="CAT_000146_e80">
   <num>80</num>
   <name type="author">Cherubini (L.),</name>
   <trait>
   <p>l'illustre compositeur</p>
   </trait>
   <desc>
   <term>L. a. s.</term>;<date>1836</date>,
   <measure type="length" unit="p" n="1">1 p.</measure> 
   <measure unit="f" type="format" n="8">in-8</measure>.
   <measure commodity="currency" unit="FRF" quantity="12">12</measure>
   </desc>
</item>

   \end{minted}
\end{listing}

into 

\begin{listing}[h!]
   \begin{minted}{json}
{"CAT_000146_e80_d1": {"desc": "L. a. s.; 1836, 1 p. in-8. 12",
   "price": 12.0,
   "author": "Cherubini",
   "date": "1836",
   "number_of_pages": 1.0,
   "format": 8,
   "term": 7,
   "sell_date": "1893-03"}}

   \end{minted}
\end{listing}

From \texttt{export.json}, we can proceed at the reconciliation of the catalogues entries. 

If you want to learn more about the reconciliation, visite \href{https://raw.github.com/katabase/reconciliation}{this repository}. 

If you want to query the database, don't hesitate to try our \href{https://raw.github.com/katabase/application}{application}.
\section*{Installation and use}
\addcontentsline{toc}{subsection}{Installation and use}

\begin{listing}[h!]
   \begin{minted}{bash}
* git clone https://github.com/katabase/3_TaggedData.git
* cd 3_TaggedData
* python3 -m venv my_env
* source my_env/bin/activate
* pip install -r requirements.txt
* cd script 
* python3 extractor_json.py

   \end{minted}
\end{listing}
\textbf{Note that you have to be in the folder \texttt{script}to execute \texttt{extractor\_json.py}.}

The output file, \texttt{export.json}, is in the folder \texttt{output}.
\subsection*{Unittest}
\addcontentsline{toc}{subsubsection}{Unittest}

If you want run some unittests, try in the folder \texttt{script}: 

\begin{listing}[h!]
   \begin{minted}{bash}
python3 test.py

   \end{minted}
\end{listing}
\section*{Credits}
\addcontentsline{toc}{subsection}{Credits}

* The script was created by Alexandre Bartz with the help of Matthias Gille Levenson and Simon Gabay.
* The catalogs were encoded by Lucie Rondeau du Noyer, Simon Gabay, Matthias Gille Levenson, Ljudmila Petkovic and Alexandre Bartz.
\section*{Cite this repository}
\addcontentsline{toc}{subsection}{Cite this repository}

Alexandre Bartz, Simon Gabay, Matthias Gille Levenson, Ljudmila Petkovic and Lucie Rondeau du Noyer, \_Manuscript sale catalogues\_, Neuchâtel: Université de Neuchâtel, 2020, \href{https://github.com/katabase/3\_TaggedData}{https://github.com/katabase/3\_TaggedData}.
\section*{Licence}
\addcontentsline{toc}{subsection}{Licence}

The catalogues are licensed under a \href{http://creativecommons.org/licenses/by/4.0/}{Creative Commons Attribution 4.0 International Licence} and the code is licensed under a GNU GPL-3.0 license.