\section*{Output Data - level 1}
\addcontentsline{toc}{section}{Output Data - level 1}

This repository contains digitised manuscripts sale catalogs encoded in XML-TEI at level 1.

The data have not been cleaned (\href{https://github.com/katabase/2\_CleanedData}{level 2}) or post-processed (\href{https://github.com/katabase/3\_TaggedData}{level 3}).
\section*{Description of the data}
\addcontentsline{toc}{subsection}{Description of the data}

Basic bibliographic information for each catalogue are available \href{https://github.com/katabase/1\_OutputData/blob/master/\_listDATA.csv}{here}.
\subsection*{Schema}
\addcontentsline{toc}{subsubsection}{Schema}

You can find the ODD that validates the encoding in the repository \href{https://github.com/katabase/Data\_extraction/tree/master/\_schemas}{Data\_extraction (folder \texttt{\_schemas})}.
\section*{Workflow}
\addcontentsline{toc}{subsection}{Workflow}
\subsection*{Creation of the data}
\addcontentsline{toc}{subsubsection}{Creation of the data}

The creation process is described in detail in the following \href{https://github.com/katabase/GROBID\_Dictionaries/blob/master/DOCUMENTATION.md}{repo}.
\subsection*{Cleaning the data}
\addcontentsline{toc}{subsubsection}{Cleaning the data}

Entries of catalogues look like the following:

\begin{listing}[h!]
   \begin{minted}{xml}
<item n="80" xml:id="CAT_000146_e80">
   <num>80</num>
   <name type="author">Cherubini (L.),</name>
   <trait>
   <p>l'illustre compositeur</p>
   </trait>
   <desc>L. a s.; 1836, 1 p 1 /2 in8.</desc>
   <measure commodity="currency" unit="FRF" quantity="12">12</measure>
</item>

   \end{minted}
\end{listing}

Most of the reconciliation process uses data from the \texttt{<desc\textgreater{}} element of our xml files. We therefore need to correct typos to ease further post-processing, \_e.g.\_
  * \texttt{L. a s.} -\textgreater{} \texttt{L. a. s.}
  * \texttt{in8} -\textgreater{} \texttt{in-8}
  * \texttt{1 /2} -\textgreater{} \texttt{1/2}
  * \texttt{1 p} -\textgreater{} \texttt{1 p.}

The \texttt{clean\_xml.py} script \href{https://github.com/katabase/1\_OutputData/blob/master/script/clean\_xml.py}{available here} tackles this problem.
\section*{Installation and use}
\addcontentsline{toc}{subsection}{Installation and use}

\begin{listing}[h!]
   \begin{minted}{bash}
* git clone https://github.com/katabase/1_OutputData.git
* cd 1_OutputData
* python3 -m venv my_env
* source my_env/bin/activate
* pip install -r requirements.txt
* python script/clean_xml.py -f FILENAME processes one single file
	OR
* python script/clean_xml.py -d DIRECTORY processes all the files contained in a directory

   \end{minted}
\end{listing}
\section*{Credits}
\addcontentsline{toc}{subsection}{Credits}

* The ODD was created by Lucie Rondeau du Noyer.

* \texttt{clean\_xml.py}was created by Simon Gabay.

* The catalogs were encoded by Lucie Rondeau du Noyer, Simon Gabay, Matthias Gille Levenson, Ljudmila Petkovic and Alexandre Bartz.

\section*{Cite this repository}
\addcontentsline{toc}{subsection}{Cite this repository}

Alexandre Bartz, Simon Gabay, Matthias Gille Levenson, Ljudmila Petkovic and Lucie Rondeau du Noyer, \_Manuscript sale catalogues\_, Neuchâtel: Université de Neuchâtel, 2019, \href{https://github.com/katabase/1\_OutputData}{https://github.com/katabase/1\_OutputData}.

\section*{Licence}
\addcontentsline{toc}{subsection}{Licence}

The catalogues are licensed under a \href{http://creativecommons.org/licenses/by/4.0/}{Creative Commons Attribution 4.0 International Licence} and the code is licensed under a GNU GPL-3.0 license.
