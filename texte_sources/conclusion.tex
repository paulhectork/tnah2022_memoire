%%%%%%%%%%%%%%%%%%%%%%%%%%%%%%%%%%%%%%%%%%%%%%%%%%%%%%%%%%%%%%%%%%%%%%%%
%%%%%%%%%%%%%%%%%%%%%%%%%%%%%%CONCLUSION %%%%%%%%%%%%%%%%%%%%%%%%%%%%%%%


% à propos du matching complet fait en 2
Ainsi, en s'appuyant sur une connaissance du corpus, il est possible d'approximer une véritable compréhension \enquote{humaine} d'un texte semi structuré, en utilisant uniquement des \glspl{expression régulière}. Si il n'est pas possible d'identifier le sens des différents éléments, il est possible de localiser et d'extraire les éléments signifiants. Il apparaît alors que ces éléments signifiants sont, d'une certaine manière, des formes, ou des motifs qui peuvent être détectés automatiquement; la signification peut être identifiée en fonction du type de donnée extraite (un nombre est une date). Mais, dans un corpus semi-structuré, il surtout est possible d'inférer du \enquote{sens} d'un élément à partir de sa position dans le texte. C'est dans ce cadre qu'un encodage en \xmltei{} des documents prend tout son intérêt: les différentes parties d'un texte sont balisées sémantiquement, ce qui est essentiel au traitement automatisé du texte: il devient possible de traiter le texte non pas dans son intégralité, mais au niveau d'une unité sémantique très précise, comme le nom donné à un manuscrit (le \tname{}) ou une brève description (le \ttrait{})