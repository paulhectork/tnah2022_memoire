%%%%%%%%%%%%%%%%%%%%%%%%%%%%%%%%%%%%%%%%%%%%%%%%%%%%%%%%%%%%%%%%%%%%%%%%%%
%%%%%%%%%%%%%%%%%%%%%%%%% RÉSUMÉ ET INTRODUCTION %%%%%%%%%%%%%%%%%%%%%%%%%
%%%%%%%%%%%%%%%%%%%%%%%%%%%%%%%%%%%%%%%%%%%%%%%%MMMMMMMM%%%%%%%%%%%%%%%%%%

\frontmatter
\chapter*{Résumé}
\addcontentsline{toc}{chapter}{Résumé}


\mainmatter
\chapter*{Introduction}
\addcontentsline{toc}{chapter}{Introduction}
J'ai choisi de structurer mon mémoire autour de plusieurs questions connexes, qui, à différents degrés, se retrouvent tout au long du développement:

En quoi la nature semi-structurée du corpus permet d'en automatiser le traitement? Comment produire des informations normalisées et exploitables à partir d'un corpus textuel semi-structuré ? En quoi ce traitement et la traduction des documents vers d'autres formats et d'autres médias impacte leur réception ? Quels sont les choix techniques qui influencent cette réception ?

Les deux premières questions, d'orientation plutôt technique, forment la colonne vertébrale pour le mémoire; elles lient deux aspects centraux: la nature du corpus et la manière dont sa structure permet toute la chaîne de traitement. Par \enquote{semi-structuré}, j'entends que, à un niveau distant, toutes les entrées de catalogue suivent la même structure; des séparateurs distinguent les différentes parties, et les informations sont souvent structurées de manière semblable pour chaque manuscrit vendu. Cela permet un traitement de \enquote{basse technologie} (\emph{low-tech}) en évitant d'entraîner de lourds modèles de traitement du langage naturel (ce qui aboutirait à des solutions complexes, difficiles à maintenir et à faire évoluer et relativement opaques dans leur fonctionnement). À l'inverse, un corpus semi-structuré peut être traité en déduisant une \enquote{structure abstraite}, que chaque entrée de catalogue partage. Il est alors possible de mettre en place des solutions techniques plus faciles, pour un résultat de qualité équivalente. Produire des \enquote{informations normalisées et exploitables} implique de traiter le corpus en cherchant des réponses à des questions de recherche précises -- dans le cadre de mon stage, une question centrale a été de chercher à isoler les facteurs déterminant le prix d'un manuscrit.

Les deux dernières questions, au premier abord plus théoriques, me semblent centrales, notamment à la troisième partie de ce mémoire. Numérisation, traitement informatisé et diffusion sur le web ne sont pas des opérations neutres, mais un ensemble de \enquote{traductions} des documents originels. Ces processus comportent une part de choix conscients, qu'il s'agit de mettre en avant. Par exemple, on considère que la majorité des documents vendus ont pour titre l'auteur.ice du document. Cette personne n'est cependant pas toujours mentionnée, et des documents peuvent être nommés d'après un lieu, un évènement ou un thème (la Révolution française, par exemple). Ces \enquote{traductions} des catalogues sont relativement discrètes tout au long de la chaîne de traitement (où le format dominant est la \tei{}, qui garde une relation d'équivalence avec le texte). C'est lors du  passage au site web que ce processus de traduction devient plus évident, et, potentiellement, plus problématique. On y abandonne la référence au document originel (les catalogues numérisés ne sont pas accessibles en ligne par un.e utilisateur.ice), le catalogue n'est plus la manière privilégiée d'accéder aux items vendus... De plus, la construction d'un site web implique la conception d'une interface et, dans notre cas, la production d'une série de visualisations intégrées au site. Le passage au site web remet aussi en cause la hiérarchie habituelle entre ingénierie et recherche: la conception d'un site ne répond pas à une question scientifique, mais elle soulève ses propres questions. Loin d'être anodines, ces problématiques de design déterminent la construction et la réception des savoirs. Il est donc important, je pense, de problématiser ces questions de visualisation et de design.