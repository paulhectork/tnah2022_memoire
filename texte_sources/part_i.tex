%%%%%%%%%%%%%%%%%%%%%%%%%%%%%%%%%%%%%%%%%%%%%%%%%%%%%%%%%%%%%%%%%%%%
%%%%%%%%%%%%%%%%%%%%%%%%%% PREMIÈRE PARTIE %%%%%%%%%%%%%%%%%%%%%%%%%
%%%%%%%%%%%%%%%%%%%%%%%%%%%%%%%%%%%%%%%%%%%%%%%%%%%%%%%%%%%%%%%%%%%%

\part{Du document numérisé au \xmltei: nature du corpus, structure des documents et méthode de production des données}
\chapter{Le marché des manuscrits autographes au prisme des catalogues de vente}
\chaptermark{Le marché des manuscrits autographes}
Ce chapitre présente l'objet d'étude du projet \mss{} : étudier le marché des manuscrits autographes du \scl{XIX}. parisien à partir de ses catalogues de vente et étudier la construction du canon littéraire au prisme du marché du manuscrit.

\section{Pourquoi étudier le marché des manuscrits autographes?}
Cette section porte sur l'intérêt scientifique des objets d'étude du projet (marché des manuscrits et étude de la construction du canon).


\section{La structure du corpus : périodisation, producteurs des documents et classification}
Ici est faite une présentation des documents traités dans le cadre du projet MSS. La présentation est à deux niveaux: au niveau du corpus et des catalogues. Ce chapitre s'appuie sur les mémoires effectués par d'ancien.ne.s stagiaires de \ktb{}, qui ont déjà beaucoup analysé la nature et les enjeux du corpus\footcite{rondeau_du_noyer_encoder_2019, corbieres_du_2020, janes_du_2021}.

\subsection{Le corpus de catalogues de vente de manuscrits}
Ici est présenté le corpus: nature, quantité de documents (et d'entrées individuelles), dates, différentes classifications qui peuvent être faites (revues, catalogues de ventes aux enchères ou à prix fixes...).

\subsection{Structure des catalogues}
Ici sera présentée la structure des catalogues; la structure de chaque page ne sera détaillée qu'à la partie suivante.


\chapter{Production des données: de l'OCR à la \tei{}}
\chaptermark{Production des données}
Cette partie s'attache autant à présenter le processus d'océrisation (qui est déjà bien établi et ne constitue pas le cœur de mon stage) que la structure des documents. Alors que le chapitre précédent s'intéresse aux catalogues dans leur ensemble, ici, on étudie le corpus au niveau de la page et de l'entrée individuelle. En effet, l'océrisation repose sur la segmentation, et donc sur l'établissement d'une structure \enquote{abstraite} d'une page (c'est-à-dire, d'un découpage de la page en zones).

\section{Extraire le texte des imprimés}
\subsection{Comprendre la structure du document pour préparer l'édition numérique}

\subsubsection{Appréhender la structure de la page à l'aide de SegmOnto}
La structure des catalogues est présentée au niveau de la page. L'ontologie SegmOnto\footcite{christensen_segmonto_2022} est utilisée, autant pour appréhender la structure de la page que pour exprimer cette structure de façon standardisée.

\subsubsection{Description des entrées de catalogue: préparer l'édition \tei{}}
Ici, la structure des catalogues est présentée au niveau de l'entrée, c'est à dire du lot mis en vente. C'est à autour de la structure des entrées qu'est construite l'édition \xmltei{}. On s'intéresse à la structure des entrées individuelles à deux niveaux:
\begin{itemize}
	\item Au niveau intellectuel: quelles sont les différentes parties d'une entrée (titre, description du manuscrit, prix...).
	\item Au niveau  \enquote{textuel}: quels sont les séparateurs, c'est à dire les éléments dans le texte qui permettent de séparer les pages de catalogue en entrées et les entrées en sous-éléments) correspondant à la structure intellectuelle décrite ci-dessus.
\end{itemize}

\section{L'encodage des manuscrits en \xmltei{}}
\subsection{Encoder les catalogues en \tei{}}
Ici est présentée la représentation \xmltei{} des catalogues de vente.

\subsection{L'encodage en \tei{}: un processus sélectif qui réduit les significations du texte}
Après une étape d'océrisation via \escr{}, le texte extrait des PDF peut être exporté soit en texte brut, soit en \xml{} \texttt{Page} ou \alto{}. Ces formats s'attachent à garder une relation entre le \xml{} et le document numérisé (les zones de texte sont indiquées, chaque ligne est dans une balise...). Cependant, l'unité intellectuelle centrale à la suite du projet, ce n'est pas la page numérisée, mais l'entrée de catalogue. Un format plus complexe que le \xml{} d'\escr{} est donc nécessaire. Assez logiquement, la suite du projet s'appuie sur une traduction des catalogues en \tei{}. On s'intéresse autant à la structure des documents \xml{} (quelles balises sont utilisées...) qu'à l'intérêt scientifique d'une édition numérique (balisage sémantique, possibilité de normaliser les informations grâce à des attributs).

L'édition numérique en \xmltei{} des catalogues implique une certaine perte d'informations: l'intégralité des significations contenues dans les catalogues imprimés ne peut être traduite en \tei{} (la police, ou la qualité du papier, peuvent être documentés mais ne peuvent pas être reproduites). Ce genre de perte d'information a lieu, à différents degrés, dans la plupart des éditions \tei{}: ce format n'est pas un substitut des documents originels. Dans le projet \mssktb{}, d'autres informations sont perdues: l'édition numérique n'est pas censée être une représentation exhaustive des catalogues. La \tei{} n'est pas utilisée comme un format de conservation, mais comme un format de traitement qui sera enrichi dans les différentes étapes. Afin de mesurer ce qui est conservé et ce qui est perdu du document originel, l'édition \tei{} sera analysée à la lumière de la \enquote{roue du texte} du philologue Patrick Sahle\footcite[p. 11]{sahle_digital_2016} qui modélise les significations plurielles d'un texte.

John Frow + Susan Pearce ?